% \refstepcounter{sdel}
\section{ A Scientific Deliverable 1}
For each scientific deliverable targeted in section~\ref{sec-deliverables} provide a full section with all the subsections described below.
\label{sec-production}

%\input{contents/scientific-deliverables/SD-<name>.tex}
\subsection{Requirements (± 15\% of section's words)}
Describe here all the properties that characterize the deliverables you produced. It should describe, for each main deliverable, what are the expected functional and non functional properties of the deliverables, who are the actors exploiting the deliverables. It is expected that you have at least one scientific deliverable (e.g. ``Scientific presentation of the Python programming language'', ``State of the art on quality models for human computer interaction'', \ldots.) and one technical deliverable (e.g. ``BSProSoft - A python/django web-site for IT job offers retrieval and analysis'', \ldots). 
\subsection{Design (± 30\% of section's words)}
Provide the necessary and most useful explanations on how those deliverables have been produced.
\subsection{Production (± 40\% of section's words)}
Provide descriptions of the deliverables concrete production. It must present part of the deliverable (e.g. source code extracts, scientific work extracts, \ldots) to illustrate and explain its actual production.
\subsection{Assessment (± 15\% of section's words)}
Provide any objective elements to assess that your deliverables do or do not satisfy the requirements described above. 
