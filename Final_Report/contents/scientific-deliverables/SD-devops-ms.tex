\refstepcounter{sdel}
\section{Scientific Deliverable \thesdel\ -- What is the relationship between DevOps and Microservices?}
\label{sd:devops}


\subsection{Requirements}

This scientific deliverable aims to find a relationship between the
DevOps approach and the \gls{ms} style. In the design section we are
therefore first going to introduce aspects of both DevOps and
\glspl{ms}. The production section will then draw parallels between
the presented characteristics which in turn allows us to observe how
both concepts are connected to each other. In the assessment we will
talk about threats to the validity of our discussions.

\subsection{Design}

\subsubsection{Relevant aspects of microservices}

In section \ref{sd:ms} we have introduced characteristics that
generally make up a \gls{ms}. We shall briefly touch upon some of the
ideas again and give a bit more insight that is relevant for our
comparison with DevOps.

\paragraph{Rapid provisioning}

To be able to rapidly launch a server---within a few hours for
instance---it is important to automate the process of
provisioning\footnote{this is the process of adding and configuring
tools onto a virtual machine}. To make serious use MSs, one should
strive for a fully automated process---as far as this is possible.
\cite{ms-prereq}

\paragraph{Basic monitoring}

Due to loosely-coupled services working together in production, things
are bound to go wrong in ways that are difficult to detect in test
environments. It is therefore of utmost importance to make sure
serious problems---appearing through technical as well as business
issues---are detected swiftly. \cite{ms-prereq}

\textit{Technical issues} could encompass run-time errors, or service
availability. \textit{Business issues} can be related to a drop in orders, or
problems with sending transactions through multiple services.
\cite{ms-prereq}

\paragraph{Rapid application deployment}

A pipeline comprises the building, testing and deploying of an
application which are steps that tend to take a long time to run
through. The idea is to break up these steps and the tasks within them
into stages. While it will add extra time to run through, we benefit
from from greater confidence because early stages can detect problems
that may have devastating effects later down the road. Later stages,
in turn, can make more in-depth testing to gain extra confidence in
the finial product. Stages in a pipeline can be automated or require
manual intervention and may even be parallelized---if applicable---to
speed up the process. Deployment pipelines are in fact a central part
of \gls{cicd}. \cite{pipelines}

Deployment usually involves a pipeline with various steps, each
providing some sort of validation of the product.  With many services
to manage, this pipeline should striven to be fully automated.
Achieving this implies close collaboration between
developers and operations: the \textit{DevOps Culture}. \cite{ms-prereq} 

\subsubsection{Relevant aspects of DevOps}

Let us now have a look at the DevOps culture before drawing out the
relations that exist between the \gls{ms} and DevOps philosophies.
\KB{Remove next sentence?} We will again have a look at aspects that
are relevant to us but nonetheless do not obfuscate the overall
approach that DevOps seeks to follow.

\paragraph{The DevOps approach}

Traditionally software development is broken down into distinct
categories such as requirements analysis, testing and development as
well as deployment, operations and maintenance once the product is
released. DevOps aims to merge these categories and favour
collaboration between \textbf{dev}elopment and
\textbf{op}eration\textbf{s}. \cite{devops-culture}

\paragraph{Shared responsibilities}

A side effect of the DevOps approach are the \textit{shared
responsibilities}, encouraging closer collaboration. As briefly
mentioned in the \nameref{sec:business-capabilities} section, now that
developers and operators are working under the same hood, the former
is much more inclined to adapt a mindset of finding ways to simplify
deployment and maintenance since they are closely involved with the
tasks of the operators. If the development team were to just hand over
the finished product and not worry about it anymore, they would not
see the difficulties operators might need to face in light of their
productions. \cite{devops-culture}

\paragraph{Quality of productions}

Given that each party is more involved in each other's tasks, this
leads to better understanding of the processes and issues which
developers and operators need to tackle. This allows for much more
efficient problem solving and leads to teams needing to value building
quality into the development process. The ultimate goal is to automate the
deployments and speeding up the testing cycle which results in greater
ease of putting code into production. \cite{devops-culture}

\paragraph{Automation}

Automation is an important factor in all of the above. For one,
automated tasks---such as testing, configuration and deployment---free
staff of these burdens and reduce human error. Another point is that
the automation scripts essentially serve as useful and up-to-date
documentation of the system. Should, for example, a developer or
operator want to inspect or change how a server is configured, they
know where to look. \cite{devops-culture}

\subsection{Production}

Given the above points and a few of the characteristics of MSs, one
could derive the implication that DevOps is an inherent aspect
of the microservice philosophy. The two points that lead
us to this assumption were the \nameref{sec:business-capabilities} and
\nameref{sec:infrastructure-automation} characteristics of \glspl{ms}.

The infrastructure automation becomes increasingly relevant if our
network of MSs expands. Managing only a handful of services by hand
can still be feasible, but if we have a few dozens or more services
this quickly becomes a virtually impossible task as we would need to
take care of each of the countless services individually. As for the
DevOps side of things, in order to increase production speed the
automation is an equally important factor. Not only does it free the
staff of work, but it also reduces human error---which seems likely to
seep in when managing lots of services in parallel. These viewpoints
on automation allow us to draw an assumption on how the DevOps
approach is vital in order to make working with microservices a
feasible task.

The second point is that the automation scripts serve as useful and
up-to-date documentation of the system. This is very useful in that
both developers and operators can inspect these scripts and be on the
same page with regards to how the system is built and works. This
directly leads us to the MS characteristic of \textit{organizing
around business capabilities}. The fact that we do not split our teams
according to application layers---as is usually the case for
monoliths---but rather according to functionalities greatly favours
collaboration of each application layer's expert. The automation
scripts coming forth through the DevOps aspect serve in this regard as
a bridge to ease this collaboration among team members and thus help
further boost and encourage the idea of organising in terms of
functionalities rather than isolated application layers.

\subsection{Assessment}

Possible threats to validity of the above discussion and assumptions
stem from the fact that they were drawn with disregard to the
technologies. Merely the researched material presented until this
point were taken into consideration.  Further, we did not have the
opportunity to experiment with this organisation of teams in practice,
which would surely provide a much better insight and grounds for our
assumptions.  That said, given more expertise on how \glspl{ms} and
DevOps work in practice might give a more precise and elaborate
answer.

% \KB{Threats to validity of the discussion in production section}

% \KB{Highlight elements pointing to TD}
