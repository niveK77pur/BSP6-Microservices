% {\color{gray}
% This document is a template for the scientific and technical (S\&T for short) report that is to be delivered by any BiCS student at the end of each Bachelor Semester Project (BSP). The Latex source files are available at:\\
% \href{https://github.com/nicolasguelfi/lu.uni.course.bics.global}{{\underline{\textbf{https://github.com/nicolasguelfi/lu.uni.course.bics.global}}}}\\
  
% This template is to be used using the Latex document preparation system or using any document preparation system. The whole document should be in between 6000 to 8000 words~\footnote{i.e. approximately 12 to 16 pages double columns} (excluding the annexes) and the proportions must be preserved. The other documents to be delivered (summaries, \ldots) should have their format adapted from this template.
% }

In the domain of software engineering, microservices are an emanate
approach for building applications. Its benefits are numerous and
target the long-term development of a product---most notably
\emph{scalability} to meet growing needs and \emph{maintenance} over the whole
life-cycle.

During this project we inspected what a microservice is and how DevOps
plays a key role in this new architectural style, the insights of
which are presented in this report.

The \emph{Energy4Life} monolithic application served as our use case
for creating a microservice. The latter is an independent
service---meaning that it should be deployable independently from
Energy4Life or any other service---that displays data from the former.
Our intention was for it to portray a proof of concept\footnote{If
needed, our microservice may be extended or re-purposed to be more
useful to the Energy4Life ecosystem.} showing that you can indeed deploy an
microservice adjacent to a monolith.  We will present the whole
process of creating this service along with the problems we faced and
how we circumvented them.
