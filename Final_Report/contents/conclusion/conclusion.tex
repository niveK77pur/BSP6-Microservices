% {\color{gray}
\section{Conclusion}

Creating this \gls{ms} was quite a challenge, mainly due to the many
unforeseen issues with the various tools. It is hard to say if a lot
of these issues could have been avoided by simply using a different
programming language than Python. It seemed like simplest for our use
case---especially since it is the language I am most comfortable
with---but given how the original \gls{e4l} code was written in Java
and did not seem to have faced similar problems, there is reason to
suspect the Python libraries we used had some trouble coping with our
setting.

Overlooking the numerous roadblocks, the process of creating the
\gls{ms} was rather straightforward. This was further boosted by the
clear boundaries between the monolith and the service, which avoided
us from having to dive in to the \gls{e4l} code in order to integrate
our functionality. We were therefore able to implement our service
without too much regard to the original code base. The end result of
having an automated pipeline further eases the production greatly---now
modifications will automatically be deployed without manual
intervention.

One should note however, that we worked in a very simplified and
almost \emph{sandbox-like} setting with only a single \gls{ms} coupled to
a monolith. We did not have to worry about managing multiple
\glspl{ms} and deal with all the complexities that stem from this. It
would thus be interesting to further investigate how \glspl{ms} would
behave in a more real setting where a lot of interplay between a
greater network of services needs to be taken care of.

% The conclusion goes here.
% }
