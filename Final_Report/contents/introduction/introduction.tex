\section{Introduction}% (± 5\% of total words)}

Traditionally, when developing software the most common way to
approach it is by making a monolithic\footnote{The application is
built using a single unit, thus everything ends up and runs in a
single executable.} application.
\cite{ms-challenges} This entails that the developer teams are
organised according to application layers
\cite{ms-definition,ms-challenges}---for instance\footnote{See section
\nameref{sec:business-capabilities}} one party will be responsible
solely for the database, yet another will take care of the actual
server capabilities (i.e. the back-end) and the last one will handle
the front-end which the end-user of the product is presented with.

However, this way of structuring the application and its development
has its downsides. A survey conducted on \gls{sos} architecting
\cite{sos-survey} revealed that there were a notable amount of
occurrences where teams from different disciplines made independent
decisions---which could be problematic for monolithic architectures in
particular. As a matter of fact, the results suggested that
integration problems arise more frequently if separate teams---or
separate groups within a same team---make
independent architectural decisions. \cite{sos-survey} Beyond this,
there is generally a lack of flexibility and scalability that arises
through the monolithic architecture. \cite{ms-challenges}

The \gls{ms} architecture aims to tackle these issues
\cite{ms-challenges} by favoring a very different approach to
development---which will be discussed in section \ref{sd:ms}.
Essentially, a single application is structured as a collection of small services
each of them running independently and inter-communicating using light
mechanisms.  \cite{ms-arch-study}

Prominent benefits brought by the \gls{ms} architecture are the
scalability of a product and the delegation of responsibilities to
independent teams\footnote{We also refer to it as
\nameref{sec:business-capabilities} \cite{ms-definition}}. \cite{ms-migration}
DevOps in an enabler which allows us to exploit
the benefits brought by \glspl{ms} to the fullest.
One should not hide the fact however,
that \gls{ms} are generally perceived to be quite difficult to deal
with. Migrating existing software to \glspl{ms} requires more
experienced developers and overall, this architectural style brings
its own set of challenges due to its inherent complexity.
\cite{ms-pains-gains} All things considered, the rather large
investments that need to be made by companies pay off rather quickly
within a span of one to three years which is directly attributed to
the long-term effects of \glspl{ms}---most notably the maintainability
and scalability lead to reduced complexity for an evolving product.
\cite{ms-migration}

Over the course of this \gls{bsp} we attempted not to restructure an
existing monolithic application into \glspl{ms} but rather to create
an independent service that will live along side the monolith. We used
the \gls{e4l} application for our case study where the adjacent
\gls{ms} would display arbitrary information contained in the
application.

Our scientific productions count the inspection of what a \gls{ms} is,
followed by how DevOps plays a role in this architectural style. The
technical production will tackle the whole process we went through in
order to create our \gls{ms} in question.

% {\color{gray}

% This paper presents the bachelor semester project made by Motivated Student together with Motivated Tutor as his motivated tutor.
% It presents the scientific and technical dimensions of the work done. All the words written here have been newly created by the authors and if some sequence of words or any graphic information created by others are included then it is explicitly indicated the original reference to the work reused. 

% This report separates explicitly the scientific work from the technical one. In deed each BSP must cover those two dimensions with a constrained balance  (cf. \cite{bics-bsp-reference-document}). Thus it is up to the Motivated Tutor and Motivated Student to ensure that the deliverables belonging to each dimension are clearly stated. As an example, a project whose title would be ``A multi-user game for multi-touch devices'' could define as scientific~\cite{armstrong2017guidelinesforscience} deliverables the following ones:
% \begin{itemize}
%   \item Study of concurrency models and their implementation
%   \item Study of ergonomics in human-computer interaction
% \end{itemize}

% The length of the report should be from 6000 to 8000 words excluding images and annexes. The sections presenting the technical and scientific deliverables represent ± 80\% of total words of the report.
% }
